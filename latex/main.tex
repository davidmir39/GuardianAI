\documentclass[12pt,a4paper]{report}

\usepackage[utf8]{inputenc}
\usepackage[spanish]{babel}
\usepackage{graphicx}
\usepackage{geometry}
\geometry{margin=2.5cm}
\usepackage{hyperref}
\usepackage{titlesec}
\usepackage{booktabs}
\usepackage{amsmath}

\titleformat{\chapter}[display]
  {\normalfont\huge\bfseries}{\chaptertitlename\ \thechapter}{20pt}{\Huge}
\titlespacing*{\chapter}{0pt}{-20pt}{40pt}

\begin{document}

\begin{titlepage}
    \centering
    \includegraphics[width=0.3\textwidth]{uclm.png}
    \vfill
    {\Large \textbf{Universidad de Castilla-La Mancha}} \\
    {\large Escuela Superior de Informática} \\
    \vspace{1cm}
    {\Large \textbf{Desarrollo e Integración de Servicios de IA}} \\
    \vspace{1.5cm}
    {\huge \textbf{GuardianAI: Fraud detection in Transactions}} \\
    \vfill
    {\large \textbf{Autores:}} \\
    {\large David Miguel Miranda Rodríguez} \\
    {\large Alejandro Córcoles Roldán} \\
    {\large Gonzalo Torres Aparicio} \\
    {\large Antonio César García Fuentes} \\
    \vfill
    {\large 2026}
\end{titlepage}

\tableofcontents
\listoffigures
\listoftables

% --- CAPÍTULO 1 ---
\chapter{Hito 1: Definición del problema y determinación del alcance}

\section{Contexto y definición del problema}
En el contexto de la seguridad y supervisión inteligente, el proyecto \textbf{GuardianAI} surge de la necesidad de...

\section{Objetivos generales y específicos}
El objetivo principal es diseñar, implementar y validar un sistema de IA capaz de...

\section{Determinación del alcance}

\subsection{Planificación}
Se estima una duración inicial de 12 semanas para las fases de desarrollo y despliegue.

\subsection{Recursos Humanos}
El equipo técnico se divide en los siguientes roles:
\begin{itemize}
    \item \textbf{ML Engineer:} Responsable del diseño de modelos.
    \item \textbf{Data Analyst:} Encargado del procesamiento de datos.
    \item \textbf{Domain Expert:} Especialista en el área de aplicación.
\end{itemize}

\subsection{Viabilidad. Estudio del ROI}
Para calcular el retorno de inversión, se utiliza la siguiente fórmula:
\begin{equation}
ROI = \frac{\text{Beneficio} - \text{Coste}}{\text{Coste}}
\end{equation}

Considerando un coste de 18.000€ y un beneficio proyectado de 24.000€, obtenemos un $ROI = 0,33$ (33\%).

\subsection{Valor}
El sistema aporta valor en cuatro dimensiones clave:
\begin{enumerate}
    \item \textbf{Valor Técnico:} Optimización de procesos.
    \item \textbf{Valor Social:} Transparencia y ética.
    \item \textbf{Valor Organizativo:} Escalabilidad.
    \item \textbf{Valor Tecnológico:} Innovación en modelos bioinspirados.
\end{enumerate}

% --- REFERENCIAS ---
\begin{thebibliography}{9}
\end{thebibliography}

\end{document}
